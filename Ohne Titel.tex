\documentclass[aspectratio=169,10pt]{beamer}

% ==================================================
% Hochschule Aalen Präsentationsvorlage (angepasst)
\documentclass[aspectratio=169,10pt]{beamer}

% ==================================================
% Hochschule Aalen Präsentationsvorlage (angepasst)
% ==================================================
\usetheme{Madrid}
\usecolortheme{dolphin}

% Schrift & Sprache
\usepackage[utf8]{inputenc}
\usepackage[T1]{fontenc}
\usepackage[ngerman]{babel}
\usepackage{lmodern}
\usepackage{graphicx}
\usepackage{booktabs}
\usepackage{hyperref}
% Priorisiere gängige Grafikformate (vermeidet Suche nach .bb-Dateien bei pdfLaTeX)
\DeclareGraphicsExtensions{.pdf,.png,.jpg}
\usepackage{listings}
\usepackage{xcolor}

% Code-Layout
\lstset{
  basicstyle=\ttfamily\footnotesize,
  keywordstyle=\color{blue},
  commentstyle=\color{gray},
  showstringspaces=false,
  frame=single,
  breaklines=true
}

% Farben Hochschule Aalen
\definecolor{hsaaBlue}{RGB}{0,70,135}
\definecolor{hsaaGray}{RGB}{100,100,100}

\setbeamercolor{structure}{fg=hsaaBlue}
\setbeamercolor{title}{fg=white,bg=hsaaBlue}
\setbeamercolor{frametitle}{fg=white,bg=hsaaBlue}
\setbeamercolor{footline}{fg=white,bg=hsaaBlue}

% Titelinformationen
	itle[Strategized Locking Pattern]{Strategized Locking Pattern}
\subtitle{Pattern-Oriented Software Architecture, Vol. 2 (POSA2)}
\author[Florian Merlau]{Florian Merlau}
\institute[Hochschule Aalen]{Hochschule Aalen\\Master Informatik -- Advanced Software Quality (WiSe 25/26)}
\date{\today}

% Logo (für alle Folien oben rechts) — nutzt vorhandene Dateien falls existent
\IfFileExists{hs-aalen-logo.png}{\logo{\includegraphics[height=1.0cm]{hs-aalen-logo.png}}}{\IfFileExists{aalen_logo_placeholder.png}{\logo{\includegraphics[height=1.0cm]{aalen_logo_placeholder.png}}}{}}

% HEADLINE: Logo oben rechts + dünner blauer Strich darunter
\setbeamertemplate{headline}{%
  % Raum für das Logo (rechts)
  \begin{beamercolorbox}[wd=\paperwidth,ht=1.0cm,dp=0ex,sep=0pt]{}
    \hfill%
    \raisebox{0.15cm}{\usebeamertemplate{logo}}\hspace{0.5cm}
  \end{beamercolorbox}%
  % dünner blauer Strich unterhalb der Kopfzeile
  \begin{beamercolorbox}[wd=\paperwidth,ht=0.12cm,dp=0ex]{footline}%
  \end{beamercolorbox}%
}

% FOOTLINE: durchgehender blauer Balken + Zeile mit Autor/Foliennummer
\setbeamertemplate{footline}{%
  % Blauer Balken (durchgehend)
  \begin{beamercolorbox}[wd=\paperwidth,ht=0.8ex,dp=0ex]{footline}
  \end{beamercolorbox}%
  % Infozeile über dem Balken (Autor links / Seitenzahl rechts)
  \begin{beamercolorbox}[wd=\paperwidth,ht=2.2ex,dp=1ex,left]{author in head/foot}
    \hspace*{2ex}\usebeamerfont{author in head/foot}\insertshortauthor
    \hfill\usebeamerfont{date in head/foot}\insertframenumber{} / \inserttotalframenumber\hspace*{2ex}
  \end{beamercolorbox}%
}

\begin{document}

% Titelseite
\begin{frame}
  	itlepage
\end{frame}

% Gliederung
\begin{frame}{Agenda}
  	ableofcontents
\end{frame}

% Einleitung & Motivation
\section{Einführung \& Motivation}
\begin{frame}{Motivation}
  \begin{itemize}
    \item Nebenläufige Software ist fehleranfällig: \textit{Race Conditions, Deadlocks}
    \item Feste Locking-Strategien sind unflexibel und schwer wartbar
    \item Ziel: Wiederverwendbare Komponenten mit austauschbaren Synchronisationsmechanismen
  \end{itemize}
\end{frame}

\begin{frame}{Problem klassischer Ansätze}
  \begin{itemize}
    \item Hard-coded Locks $\rightarrow$ geringe Flexibilität
    \item Mehrere Varianten desselben Codes für unterschiedliche Threading-Modelle
    \item Schwierige Wartung, fehleranfällig, Performance-Verlust
  \end{itemize}
\end{frame}

% Patternbeschreibung
\section{Patternbeschreibung}
\begin{frame}{Intent und Kontext}
  	extbf{Intent:}\\
  	extit{Strategize a component’s synchronization to increase its flexibility and reusability
  without degrading performance or maintainability.} (Schmidt, 1998)

  \vspace{1em}
  	extbf{Kontext:}
  \begin{itemize}
    \item Wiederverwendbare Komponenten, die in verschiedenen Concurrency-Umgebungen laufen müssen
    \item Locking soll konfigurierbar oder „pluggable" sein
  \end{itemize}
\end{frame}

\begin{frame}{Problem \& Forces}
  \begin{itemize}
    \item Unterstützung verschiedener Locking-Strategien (Mutex, RW-Lock, Null-Mutex)
    \item Anforderungen: Performance, Wartbarkeit, Portabilität
    \item Trade-off: Compile-Time Templates vs. Run-Time Polymorphie
  \end{itemize}
\end{frame}

\begin{frame}{Lösung – Strategized Locking}
  \begin{itemize}
    \item Synchronisation wird ausgelagert in eine "Strategie"-Klasse
    \item Zwei Varianten:
      \begin{enumerate}
        \item \textbf{Polymorph} – Lock-Typ wird zur Laufzeit gewählt
        \item \textbf{Template-basiert} – Lock-Typ wird beim Kompilieren festgelegt
      \end{enumerate}
    \item Nutzung des \textbf{Scoped Locking Idioms} (RAII)
  \end{itemize}
\end{frame}

% Codebeispiel und Diagramm
\section{Codebeispiel und Diagramm}
\begin{frame}[fragile]{Codebeispiel (C++)}
\begin{lstlisting}[language=C++]
template <class LOCK>
class File_Cache {
public:
  const char* find(const char* path) {
    Guard<LOCK> guard(lock_);
    // Zugriff auf Cache...
    return nullptr;
  }
private:
  LOCK lock_;
};
\end{lstlisting}
\end{frame}

\begin{frame}{Diagramm – Strategized Locking}
  \begin{center}
    \includegraphics[width=0.92\linewidth,height=0.84\textheight,keepaspectratio]{diagram-placeholder.pdf}
  \end{center}
\end{frame}

\begin{frame}{Vergleich der Strategien}
  \begin{tabular}{|l|l|l|l|}
  \hline
  	extbf{Strategie} & \textbf{Beschreibung} & \textbf{Vorteil} & \textbf{Nachteil} \\
  \hline
  Null\_Mutex & Kein Lock (single-threaded) & Schnell & Unsicher bei Threads \\
  Thread\_Mutex & Standard Lock & Robust & Langsam bei hoher Parallelität \\
  RW\_Lock & Parallel Lesen, exklusives Schreiben & Effizient bei vielen Reads & Komplexer \\
  \hline
  \end{tabular}
\end{frame}

% Verwandte Patterns
\section{Verwandte Patterns}
\begin{frame}{Thread-safe Interface Pattern}
  \begin{itemize}
    \item Verhindert \textit{Self-Deadlocks} durch klare Trennung
    \item \textbf{Interface methods check} – öffentliche Methoden sichern Zugriff
    \item \textbf{Implementation methods trust} – interne Methoden vertrauen auf bestehenden Lock
  \end{itemize}
\end{frame}

\begin{frame}{Scoped Locking Idiom}
  \begin{itemize}
    \item RAII-Prinzip: Lock wird beim Eintritt erworben, beim Verlassen freigegeben
  \end{itemize}
  \vspace{1em}
  \begin{lstlisting}[language=C++]
template<class LOCK>
class Guard {
public:
  Guard(LOCK &l): lock(l) { lock.acquire(); }
  ~Guard() { lock.release(); }
private:
  LOCK &lock;
};
  \end{lstlisting}
\end{frame}

% Bewertung & Fazit
\section{Bewertung \& Fazit}
\begin{frame}{Vor- und Nachteile}
  	extbf{Vorteile:}
  \begin{itemize}
    \item Hohe Flexibilität und Wiederverwendbarkeit
    \item Wartungsfreundlich (eine zentrale Implementierung)
    \item Einfache Performance-Anpassung
  \end{itemize}

  \vspace{1em}
  	extbf{Nachteile:}
  \begin{itemize}
    \item Template-Komplexität / Sichtbarkeit der Strategien
    \item Potenziell höhere Kompilierzeit
  \end{itemize}
\end{frame}

\begin{frame}{Bekannte Anwendungen}
  \begin{itemize}
    \item \textbf{ACE Framework (Adaptive Communication Environment)}
    \item \textbf{Booch Components}
    \item Moderne Äquivalente: \texttt{std::lock\_guard}, Java \texttt{synchronized}
  \end{itemize}
\end{frame}

\begin{frame}{Fazit}
  \begin{itemize}
    \item Strategized Locking = „pluggable synchronization"
    \item Klare Trennung von Funktionalität und Synchronisation
    \item Grundlage vieler moderner Concurrency-Patterns
  \end{itemize}
\end{frame}

\begin{frame}{Diskussion}
  \begin{itemize}
    \item Wann lohnt sich Strategized Locking in modernen Systemen?
    \item Welche Alternativen gibt es in Java, C++, Rust?
  \end{itemize}
\end{frame}

% Quellen
\begin{frame}[allowframebreaks]{Quellen}
  	iny
  \begin{thebibliography}{9}
  \bibitem{schmidt1998}
    Douglas C. Schmidt,
    \emph{Strategized Locking, Thread-safe Interface, and Scoped Locking},
    C++ Report, Washington University, 1998.
  \bibitem{buschmann2000}
    Buschmann, F., Schmidt, D. C., et al.
    \emph{Pattern-Oriented Software Architecture, Volume 2: Patterns for Concurrent and Networked Objects.}
    Wiley, 2000.
  \end{thebibliography}
\end{frame}

\end{document}
    \item Potenziell höhere Kompilierzeit
