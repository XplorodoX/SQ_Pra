\documentclass[aspectratio=169,10pt]{beamer}

% ==================================================
% Hochschule Aalen Präsentationsvorlage (1:1 Design)
% ==================================================
\usetheme{default}
\usecolortheme{default}

% --------------------------------------------------
% Farben (Corporate Design)
% --------------------------------------------------
\definecolor{hsaaBlue}{RGB}{0,70,135}
\definecolor{hsaaGray}{RGB}{80,80,80}

% --------------------------------------------------
% Pakete
% --------------------------------------------------
\usepackage[utf8]{inputenc}
\usepackage[T1]{fontenc}
\usepackage[ngerman]{babel}
\usepackage{lmodern}
\usepackage{graphicx}
\usepackage{ragged2e}
\usepackage{booktabs}
\usepackage{xcolor}
\usepackage{hyperref}

% --------------------------------------------------
% Grundlayout
% --------------------------------------------------
\setbeamercolor{normal text}{fg=black,bg=white}
\setbeamercolor{frametitle}{fg=hsaaBlue,bg=white}
\setbeamercolor{structure}{fg=hsaaBlue}
\setbeamertemplate{navigation symbols}{}

% ==================================================
% Kopfzeile: Logo oben rechts + dünner blauer Strich darunter
% ==================================================
\setbeamertemplate{headline}{%
  % Logo oben rechts in Box mit ausreichender Höhe
  \begin{beamercolorbox}[wd=\paperwidth,ht=1.2cm,dp=0.0cm]{}%
    \vspace{0.1cm}
    \hfill
    % Bildhöhe auf 1cm setzen, so ist es zuverlässig sichtbar
    \includegraphics[height=1cm]{Hochschule-aalen.svg.png}\hspace{0.5cm}
  \end{beamercolorbox}%
  % Dünner blauer Strich direkt unter dem Logo, über gesamte Breite
  \begin{beamercolorbox}[wd=\paperwidth,ht=1pt,dp=0pt]{frametitle}%
    \color{hsaaBlue}\rule{\paperwidth}{1pt}%
  \end{beamercolorbox}%
}

% ==================================================
% Fußzeile: dünner blauer Balken über gesamte Breite (sichtbar auf allen Folien)
% enthält Datum / Institut, Autor / Seitenzahl in kleiner Schrift
% ==================================================
\setbeamercolor{footline}{bg=hsaaBlue,fg=white}
\setbeamertemplate{footline}{%
  % Breiterer blauer Balken mit weißer Beschriftung (zentriert vertikal)
  % Verwende ein HBox to \paperwidth und manuelle seitliche Abstände,
  % damit die Seitenzahl zuverlässig am rechten Rand innerhalb des Balkens sitzt.
  \begin{beamercolorbox}[wd=\paperwidth,ht=3.6ex,dp=1.4ex,leftskip=0pt,rightskip=0pt]{footline}%
    \leavevmode\hbox to\paperwidth{%
      \hspace{0.6cm}% left padding
      {\scriptsize\color{white}\insertdate\hfill\insertshortinstitute, \insertshortauthor}%
      \hfill%
      {\scriptsize\color{white}\insertframenumber/\inserttotalframenumber}%
      \hspace{0.6cm}% right padding
    }%
  \end{beamercolorbox}%
}

% ==================================================
% Titelinformationen
% ==================================================
\title[Thema]{Thema}
\author[Merlau]{Florian Merlau}
\institute[Hochschule Aalen]{Hochschule Aalen}
\date{10/27/25}

% ==================================================
% Dokument
% ==================================================
\begin{document}

% --------------------------------------------------
% Titelfolie im Original-Look
% --------------------------------------------------
\begin{frame}

\vspace{1cm}
\begin{flushleft}
    {\usebeamercolor[fg]{frametitle}\bfseries\LARGE Thema}
\end{flushleft}

\vspace{1cm}
\justifying
\small
Projektpräsentation Team Titel\\
Projektmanagement / Qualität und Nachhaltigkeit (…)
Prof. Dr. Ulrich Holzbaur\\[0.5em]
In Kooperation mit Stabsstelle xy der Hochschule, Amt für xy der Stadt, Her/Frau zz\\[0.5em]
Teamleiter: …\\[0.5em]
Team: \\
… W1, \\
… W1, \\
… MIM, \\
…\\[0.5em]
26. Juni 2009

\end{frame}

% --------------------------------------------------
% Beispiel Folie (mit Kopf/Fußlayout)
% --------------------------------------------------
\begin{frame}{Executive Summary – project description}
\begin{itemize}
    \item Name and short description
    \item Vision and outcome
    \item Partners and Team
\end{itemize}
\end{frame}

\end{document}
\documentclass[aspectratio=169,10pt]{beamer}

% ==================================================
% Hochschule Aalen Präsentationsvorlage (1:1 Design)
% ==================================================
\usetheme{default}
\usecolortheme{default}

% --------------------------------------------------
% Farben (Corporate Design)
% --------------------------------------------------
\definecolor{hsaaBlue}{RGB}{0,70,135}
\definecolor{hsaaGray}{RGB}{80,80,80}

% --------------------------------------------------
% Pakete
% --------------------------------------------------
\usepackage[utf8]{inputenc}
\usepackage[T1]{fontenc}
\usepackage[ngerman]{babel}
\usepackage{lmodern}
\usepackage{graphicx}
\usepackage{ragged2e}
\usepackage{booktabs}
\usepackage{xcolor}
\usepackage{hyperref}

% --------------------------------------------------
% Grundlayout
% --------------------------------------------------
\setbeamercolor{normal text}{fg=black,bg=white}
\setbeamercolor{frametitle}{fg=hsaaBlue,bg=white}
\setbeamercolor{structure}{fg=hsaaBlue}
\setbeamertemplate{navigation symbols}{}

% ==================================================
% Kopfzeile: Logo oben rechts + dünner blauer Strich darunter
% ==================================================
\setbeamertemplate{headline}{%
  % Logo oben rechts in Box mit ausreichender Höhe
  \begin{beamercolorbox}[wd=\paperwidth,ht=1.2cm,dp=0.0cm]{}%
    \vspace{0.1cm}
    \hfill
    % Bildhöhe auf 1cm setzen, so ist es zuverlässig sichtbar
    \includegraphics[height=1cm]{Hochschule-aalen.svg.png}\hspace{0.5cm}
  \end{beamercolorbox}%
  % Dünner blauer Strich direkt unter dem Logo, über gesamte Breite
  \begin{beamercolorbox}[wd=\paperwidth,ht=1pt,dp=0pt]{frametitle}%
    \color{hsaaBlue}\rule{\paperwidth}{1pt}%
  \end{beamercolorbox}%
}

% ==================================================
% Fußzeile: dünner blauer Balken über gesamte Breite (sichtbar auf allen Folien)
% enthält Datum / Institut, Autor / Seitenzahl in kleiner Schrift
% ==================================================
\setbeamercolor{footline}{bg=hsaaBlue,fg=white}
\setbeamertemplate{footline}{%
  % Breiterer blauer Balken mit weißer Beschriftung (zentriert vertikal)
  % Verwende ein HBox to \paperwidth und manuelle seitliche Abstände,
  % damit die Seitenzahl zuverlässig am rechten Rand innerhalb des Balkens sitzt.
  \begin{beamercolorbox}[wd=\paperwidth,ht=3.6ex,dp=1.4ex,leftskip=0pt,rightskip=0pt]{footline}%
    \leavevmode\hbox to\paperwidth{%
      \hspace{0.6cm}% left padding
      {\scriptsize\color{white}\insertdate\hfill\insertshortinstitute, \insertshortauthor}%
      \hfill%
      {\scriptsize\color{white}\insertframenumber/\inserttotalframenumber}%
      \hspace{0.6cm}% right padding
    }%
  \end{beamercolorbox}%
}

% ==================================================
% Titelinformationen
% ==================================================
	itle[Thema]{Thema}
\author[Merlau]{Florian Merlau}
\institute[Hochschule Aalen]{Hochschule Aalen}
\date{10/27/25}

% ==================================================
% Dokument
% ==================================================
\begin{document}

% --------------------------------------------------
% Titelfolie im Original-Look
% --------------------------------------------------
\begin{frame}

\vspace{1cm}
\begin{flushleft}
    {\usebeamercolor[fg]{frametitle}\bfseries\LARGE Thema}
\end{flushleft}

\vspace{1cm}
\justifying
\small
Projektpräsentation Team Titel\\
Projektmanagement / Qualität und Nachhaltigkeit (…)\newline
Prof. Dr. Ulrich Holzbaur\\[0.5em]
In Kooperation mit Stabsstelle xy der Hochschule, Amt für xy der Stadt, Her/Frau zz\\[0.5em]
Teamleiter: …\\[0.5em]
Team: \\
… W1, \\
… W1, \\
… MIM, \\
…\\[0.5em]
26. Juni 2009

\end{frame}

% --------------------------------------------------
% Beispiel Folie (mit Kopf/Fußlayout)
% --------------------------------------------------
\begin{frame}{Executive Summary – project description}
\begin{itemize}
    \item Name and short description
    \item Vision and outcome
    \item Partners and Team
\end{itemize}
\end{frame}

\end{document}
